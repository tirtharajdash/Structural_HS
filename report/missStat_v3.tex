\documentclass[11pt,a4paper,draft]{article}
\usepackage[a4paper, total={7in, 9in}]{geometry}
\usepackage[latin1]{inputenc}
\usepackage{amsmath}
\usepackage{amsfonts}
\usepackage{amssymb}
\usepackage{graphicx,color,xcolor}
\usepackage{enumerate}
\usepackage{url}

\title{Simulation experiments for hide-and-seek with different seeker distribution update strategies}
\author{Tirtharaj Dash}
\date{March 10, 2020}

\begin{document}

\maketitle

\section{Setup}

\noindent
We perform some simulation experiments for Hide-and-Seek with three different \textbf{seeker distribution update strategies}:
\begin{enumerate}[(1)]
	\item\label{snoupd} \textbf{No update:} No update of the seeker distribution (leads to hide-and-seek with replacement results).
	\item\label{supdwor} \textbf{Uniform update:} Open a box, distribute its probability mass to every other unopened boxes, make its probability 0 (hide-and-seek without replacement).
	\item\label{supdhc} \textbf{Hot-cold update:} The seeker updates its probability distribution based on whether it opened a cold box or a hot box. A cold box is a box for which the performance of a box ($perf$) is less than the cold threshold ($\theta_c$) and a hot box is a box with performance greater than a hot threshold ($\theta_h$). We devise the following procedure for update:
	\begin{enumerate}[1.]
		\item Open a box $i$
		\item If $perf(i) \geq \theta_h$: distribute its probability mass to all the unopened boxes in its neighbors. 
		\item If $perf(i) \leq \theta_c$: distribute its probability mass to all the unopened boxes except its neighbors.
		\item If none of 2 or 3: distribute its probability mass to all the unopened boxes.
		\item Repeat 1--4 until the hider is found.
	\end{enumerate}
	\item\label{supdloc} \textbf{Localised-sampling update:} Open a box. If the hider is not found in that box, open its unopened neighbors starightaway and check if the hider is found there. If hider is not found, then distribute the probability mass of all the boxes (the sampled box and its neighbors) to all unopened boxes equally. 
\end{enumerate}

\noindent
The seeker update strategy (\ref{supdhc}) requires that the hider distribution falls into some continuity assumption. That is: the probability mass of a neighborhood of a box are in monotonic relationship to the probability of that box. This is a realistic demand and clauses which are related with each other are monotonic in their performance in some fashion. We construct such a hider distribution ($H$) with the following code\footnote{\url{https://github.com/tirtharajdash/multimodalGaussianDistro}}.

\section{Experiments} \label{base_expt}

\begin{description}
	\item[Parameter setting] The experiments are performed for number of boxes $n = \{1000, 2000, 3000\}$. The maximum hiding trials is set at $1000$. We call it a \textbf{failure}, if the hider is not found within $n$ searches by using the seeker distribution. The neighborhood size ($nbd$) is varied as \{1,2,3\}. For all the experiments reported here, we define performance of a box by: $perf(i) = \frac{h_i}{\max(h_1,\ldots, h_n)}$, where $H = \{h_1,\ldots, h_n\}$ is the hider distribution. The thresholds are fixed at $\theta_h = 0.80$ and $\theta_c = 0.4$. The proportion of boxes that have high probability mass (spikes in $H$) is fixed at 10\%. For the seeker update stragegy described in \ref{supdloc} requires a neighborhood size ($lnbd$). This is varied as \{1,2,3\}.
	\label{setting:base}

	\item[Results] The mean and standard deviations of misses are calculated only for successful runs i.e. the hider was found by the seeker within maximum of $n$ look-ups. Otherwise, it was treated as a failure and this result was not included for statistics. Below, we report results for each seeker update strategies.
	\begin{figure}[!h]
	\centering
	\begin{tabular}{llll}
		\hline \hline 
		choiceUpdS & SuccessRate & mean(misses) & sd(misses) \\
		\hline \hline 
		\multicolumn{4}{c}{$n = 1000$} \\ 
		\hline 
		1 &  0.622 & 439.738 & 279.218 \\
		2 &  1.000 & 499.379 & 279.060 \\
		3 ($nbd=1$) & 1.000 & 486.041 & 271.251 \\
		3 ($nbd=2$) & 1.000 & 458.027 & 269.987 \\
		3 ($nbd=3$) & 1.000 & 479.270 & 279.110 \\
		4 ($lnbd=1$) & 1.000 & 713.899 & 281.582 \\
		4 ($lnbd=2$) & 1.000 & 783.193 & 255.153 \\
		4 ($lnbd=3$) & 1.000 & 839.920 & 234.323 \\
		\hline 
		\hline 
		\multicolumn{4}{c}{$n = 2000$} \\ 
		\hline 
		1 &  0.635 & 859.123 & 571.302 \\
		2 &  1.000 & 1022.204 & 595.486 \\
		3 ($nbd=1$) & 1.000 & 963.457 & 550.133 \\
		3 ($nbd=2$) & 1.000 & 961.946 & 551.028 \\
		3 ($nbd=3$) & 1.000 & 968.335 & 564.515 \\
		4 ($lnbd=1$) & 1.000 & 1458.508 & 530.032 \\
		4 ($lnbd=2$) & 1.000 & 1597.152 & 470.655 \\
		4 ($lnbd=3$) & 1.000 & 1689.032 & 460.411 \\
		\hline 
		\hline 
		\multicolumn{4}{c}{$n = 3000$} \\ 
		\hline 
		1 &  0.652 & 1303.462 & 854.176 \\
		2 &  1.000 & 1473.717 & 865.756 \\
		3 ($nbd=1$) & 1.000 & 1436.214 & 801.100 \\
		3 ($nbd=2$) & 1.000 & 1396.195 & 840.066 \\
		3 ($nbd=2$) & 1.000 & 1464.990 & 840.695 \\
		4 ($lnbd=1$) & 1.000 & 2158.499 & 824.005 \\
		4 ($lnbd=2$) & 1.000 & 2394.354 & 743.630 \\
		4 ($lnbd=3$) & 1.000 & 2531.999 & 687.815 \\
		\hline 
		\hline 
	\end{tabular}
	\caption{Average number of misses: $H$ is a multimodal distribution with 10\% spikes, and $S$ is updated using three different update strategies (1: no update, 2: without replacement, 3: update using $\theta_h$ and $\theta_c$ using three different neighborhoods in \{1,2,3\}, 4: local sampling based update with local neighborhood in \{1,2,3\}).}
	\end{figure}

	\item[Interpretation 1] The results suggest that the threshold based seeker update strategy (\ref{supdhc}):
	\begin{itemize}
		\item Reduces the average number of misses in comparison with search without replacement for which the expected number of misses is $\frac{n-1}{2}$ (i.e. choiceUpdS$=2$ in tables).
		\item For the seeker updat type (\ref{supdhc}), best neighborhood size is found to be 2. 
		\item Increasing the neighborhood size of a box increases the average number of misses in almost all the cases.
	\end{itemize}

	\item[Interpretation 2] The localised sampling based seeker update strategy described in \ref{supdloc} has following effects on average number of misses.
	\begin{itemize}
		\item It is worse than other update strategies (1--3).
		\item Bigger neighborhood sizes have adverse effect on the avergae number of misses. This could be due to the fact that we are starting with a uniform neighborhood and we are opening opening more number of bad boxes.
	\end{itemize}
	
\end{description}


	
\end{document}
