\documentclass[11pt,a4paper,draft]{article}
\usepackage[a4paper, total={7in, 9in}]{geometry}
\usepackage[latin1]{inputenc}
\usepackage{amsmath}
\usepackage{amsfonts}
\usepackage{amssymb}
\usepackage{graphicx,color}
\usepackage{enumerate}
\usepackage{url}

\title{Simulation experiments for hide-and-seek with different seeker distribution update strategies}
\author{Tirtharaj Dash}
\date{January 18, 2020}

\begin{document}

\maketitle

\section{Setup}

\noindent
We perform some simulation experiments for Hide-and-Seek with three different \textbf{seeker distribution update strategies}:
\begin{enumerate}[(1)]
	\item\label{snoupd} \textbf{No update:} No update of the seeker distribution (leads to hide-and-seek with replacement results).
	\item\label{supdwor} \textbf{Uniform update:} Open a box, distribute its probability mass to every other unopened boxes, make its probability 0 (hide-and-seek without replacement).
	\item\label{supdhc} \textbf{Hot-cold update:} The seeker updates its probability distribution based on whether it opened a cold box or a hot box. A cold box is a box for which the performance of a box ($perf$) is less than the cold threshold ($\theta_c$) and a hot box is a box with performance greater than a hot threshold ($\theta_h$). We devise the following procedure for update:
	\begin{enumerate}[1.]
		\item Open a box $i$
		\item If $perf(i) \geq \theta_h$: distribute its probability mass to all the unopened boxes in its neighbors. 
		\item If $perf(i) \leq \theta_c$: distribute its probability mass to all the unopened boxes except its neighbors.
		\item If none of 2 or 3: distribute its probability mass to all the unopened boxes.
		\item Repeat 1--4 until the hider is found.
	\end{enumerate}
\end{enumerate}

\noindent
The seeker update strategy (\ref{supdhc}) requires that the hider distribution falls into some continuity assumption. That is: the probability mass of a neighborhood of a box are in monotonic relationship to the probability of that box. This is a realistic demand and clauses which are related with each other are monotonic in their performance in some fashion. We construct such a hider distribution ($H$) with the following code\footnote{\url{https://github.com/tirtharajdash/multimodalGaussianDistro}}.

\section{Experiments} \label{base_expt}

\begin{description}
	\item[Parameter setting] The experiments are performed for number of boxes $n = \{1000, 2000, 3000\}$. The maximum hiding trials is set at $1000$. We call it a \textbf{failure}, if the hider is not found within $n$ searches by using the seeker distribution. The neighborhood size ($nbd$) is varied as \{1,2,3\}. For all the experiments reported here, we define performance of a box by: $perf(i) = \frac{h_i}{\max(h_1,\ldots, h_n)}$, where $H = \{h_1,\ldots, h_n\}$ is the hider distribution. The thresholds are fixed at $\theta_h = 0.80$ and $\theta_c = 0.4$. The proportion of boxes that have high probability mass (spikes in $H$) are fixed at 10\%.
	\label{setting:base}

	\item[Results] The mean and standard deviations of misses are calculated only for successful runs i.e. the hider was found by the seeker within maximum of $n$ look-ups. Otherwise, it was treated as a failure and this result was not included for statistics. Below, we report results for each seeker update strategies.
	\begin{figure}[!h]
		\centering
		\begin{tabular}{llll}
			\hline \hline 
			choiceUpdS & SuccessRate & mean(misses) & sd(misses) \\
			\hline \hline 
			\multicolumn{4}{c}{$n = 1000$} \\ 
			\hline 
			1 &  0.617 & 410.929 & 278.666 \\
			2 &  1.000 & 507.848 & 284.743 \\
			3 ($nbd=1$) & 1.000 & 470.455 & 269.662 \\
			3 ($nbd=2$) & 1.000 & 481.031 & 273.743 \\
			3 ($nbd=3$) & 1.000 & 496.920 & 281.998 \\
			\hline 
			\hline 
			\multicolumn{4}{c}{$n = 2000$} \\ 
			\hline 
			1 &  0.636 & 825.852 & 570.932 \\
			2 &  1.000 & 1032.257 & 582.563 \\
			3 ($nbd=1$) & 1.000 & 941.797 & 559.658 \\
			3 ($nbd=2$) & 1.000 & 935.739 & 552.096 \\
			3 ($nbd=3$) & 1.000 & 993.009 & 565.132 \\
			\hline 
			\hline 
			\multicolumn{4}{c}{$n = 3000$} \\ 
			\hline 
			1 &  0.646 & 1290.207 & 844.649 \\
			2 &  1.000 & 1489.962 & 867.514 \\
			3 ($nbd=1$) & 1.000 & 1366.866 & 800.812 \\
			3 ($nbd=2$) & 1.000 & 1430.131 & 820.963 \\
			3 ($nbd=3$) & 1.000 & 1429.897 & 840.993 \\
			\hline 
			\hline 
		\end{tabular}
		\caption{Average number of misses for three seeker update strategies}
	\end{figure}

	\item[Interpretation] The results suggest that the seeker update strategy (\ref{supdhc}):
	\begin{itemize}
		\item reduces the average number of misses in comparison with search without replacement for which the expected number of misses is $\frac{n-1}{2}$ (i.e. choiceUpdS$=2$ in tables).
		\item Increasing the $nbd$ a box increases the average number of misses in almost all the cases.
	\end{itemize}
	
\end{description}


	
\end{document}
