\documentclass[11pt,a4paper,draft]{article}
\usepackage[a4paper, total={7in, 9.5in}]{geometry}
\usepackage[latin1]{inputenc}
\usepackage{amsmath}
\usepackage{amsfonts}
\usepackage{amssymb}
\usepackage{graphicx,color}
\usepackage{enumerate}
\title{Simulation experiments for hide-and-seek with different seeker distribution update strategies}
\author{Tirtharaj Dash}
\date{January 08, 2020}

\begin{document}

\maketitle

\section{Preliminary Setup}

\noindent
We perform some simulation experiments for Hide-and-Seek with three different \textbf{seeker distribution update strategies}:
\begin{enumerate}[(1)]
	\item \textbf{No update:} no update of the seeker distribution (leads to hide-and-seek with replacement results)
	\item \textbf{Uniform update:} open a box, distribute its probability mass to every other unopened boxes, make its probability 0 (hide-and-seek without replacement)
	\item \textbf{Hot-cold update:} open a box, check if it is hot (based on some threshold criteria), if it is hot, distribute its mass to its unopened neighbors; otherwise distribute its mass to every other unopened boxes, and make its probability 0 (hide-and-seek without replacement). We call a box $i$ hot, if the performance of the box $i$, $perf(i) \geq \theta_h$.
\end{enumerate}

\noindent
To observe the expected number of misses the \textbf{hider distribution} is varied across four different \textbf{forms}:
\begin{enumerate}[(1)]
	\item \textbf{Easy hider:} The hider distribution is easy. It translate to a single spike (degenerate distribution) such that there is absolute certainty that the hider will always hide there.
	\item \textbf{Not-so-easy hider:} The hider distibution is less easy. It translates to a distribution where there are some boxes which have masses and majority of boxes have zero probability mass. We can call it semi-degenerate hider.
	\item \textbf{Not-so-hard hider:} Here the hider distribution is not-uniform. It is a random distribution. Each box have a non-zero probability mass such that it is not a uniform distribution.
	\item \textbf{Hard hider:} The hider distribution is adversarial. This translates to a uniform hider distribution in which randomness is very high.
\end{enumerate}

\section{Base Experiments} \label{base_expt}

\begin{description}
	\item[Parameter setting] The experiments are performed for number of boxes $n = \{1000, 2000, 3000\}$. The maximum hiding trials is set at $1000$. We call it a \textbf{failure}, if the hider is not found within $n$ searches by using the seeker distribution. The threshold temperature for deciding whether a box is hot is varied as $\theta_h = \{0.50, 0.80\}$. The neighborhood size ($nbd$) is fixed at 1 i.e. a box $i$ has neighbors $i-1$ and $i+1$ except for the extreme cases like $i=1$ and $i=n$ for which neighbors are $i+1$ and $i-1$ respectively. For simulation with not-so-easy hider, we have fixed the degeneracy to 25\% i.e.  a hider distribution in which 25\% boxes have non-zero probability. For all ther experiments reported here, we define performance of a box by: $perf(i) = \frac{h_i}{\max(h_1,\ldots, h_n)}$, where $H = \{h_1,\ldots, h_n\}$ is the hider distribution. 
	\label{setting:base}

	\item[Results] The mean and standard deviations of misses are calculated only for successful runs i.e. the hider was found by the seeker within $n$ look-ups. Otherwise, it was treated as a failure and this result was not included for statistics.

	\begin{figure}[!h]
		\centering
		\begin{tabular}{lllll}
			\hline \hline 
			choiceH & choiceUpdS & SuccessRate & mean(misses) & sd(misses) \\
			\hline \hline 
			\multicolumn{5}{c}{$n = 1000$} \\ 
			\hline 
			1 & 1 &  0.632 & 430.848 & 290.980 \\
			1 & 2 &  1.000 & 507.457 & 292.641 \\
			1 & 3 &  1.000 & 492.347 & 287.209 \\
			\hline 
			2 & 1 &  0.649 & 417.556 & 276.379 \\
			2 & 2 &  1.000 & 493.118 & 292.701 \\
			2 & 3 &  1.000 & 507.946 & 294.560 \\
			\hline 
			3 & 1 &  0.608 & 437.202 & 289.004 \\
			3 & 2 &  1.000 & 492.452 & 286.495 \\
			3 & 3 &  1.000 & 494.038 & 289.880 \\
			\hline 
			4 & 1 &  0.647 & 422.825 & 280.877 \\
			4 & 2 &  1.000 & 496.385 & 291.528 \\
			4 & 3 &  1.000 & 512.331 & 285.820 \\
			\hline 
			\hline 
			\multicolumn{5}{c}{$n = 2000$} \\ 
			\hline 
			1 & 1 &  0.629 & 828.405 & 567.945 \\
			1 & 2 &  1.000 & 1041.655 & 573.419 \\
			1 & 3 &  1.000 & 1009.698 & 593.316 \\
			\hline 
			2 & 1 &  0.634 & 816.994 & 578.041 \\
			2 & 2 &  1.000 & 989.527 & 586.232 \\
			2 & 3 &  1.000 & 1002.935 & 569.378 \\
			\hline 
			3 & 1 &  0.646 & 830.492 & 565.193 \\
			3 & 2 &  1.000 & 991.515 & 577.517 \\
			3 & 3 &  1.000 & 998.342 & 580.462 \\
			\hline 
			4 & 1 &  0.641 & 857.005 & 585.562 \\
			4 & 2 &  1.000 & 979.207 & 572.873 \\
			4 & 3 &  1.000 & 974.992 & 587.949 \\
			\hline 
			\hline 
			\multicolumn{5}{c}{$n = 3000$} \\ 
			\hline 
			1 & 1 &  0.625 & 1208.155 & 842.016 \\
			1 & 2 &  1.000 & 1496.376 & 857.969 \\
			1 & 3 &  1.000 & 1462.101 & 884.993 \\
			\hline 
			2 & 1 &  0.617 & 1272.316 & 842.560 \\
			2 & 2 &  1.000 & 1516.441 & 864.525 \\
			2 & 3 &  1.000 & 1449.057 & 855.567 \\
			\hline 
			3 & 1 &  0.637 & 1251.805 & 860.314 \\
			3 & 2 &  1.000 & 1476.497 & 888.707 \\
			3 & 3 &  1.000 & 1506.054 & 887.445 \\
			\hline 
			4 & 1 &  0.620 & 1224.790 & 845.916 \\
			4 & 2 &  1.000 & 1439.318 & 852.015 \\
			4 & 3 &  1.000 & 1486.285 & 846.192 \\
			\hline 
			\hline 
		\end{tabular}
		\caption{Base setting results with $\theta_h = 0.50$, $nbd = 1$}
	\end{figure}

	\begin{figure}[!h]
		\centering
		\begin{tabular}{lllll}
			\hline \hline 
			choiceH & choiceUpdS & SuccessRate & mean(misses) & sd(misses) \\
			\hline \hline 
			\multicolumn{5}{c}{$n = 1000$} \\ 
			\hline 
			1 & 1 &  0.618 & 406.259 & 284.899 \\
			1 & 2 &  1.000 & 500.030 & 287.801 \\
			1 & 3 &  1.000 & 498.926 & 291.053 \\
			\hline 
			2 & 1 &  0.624 & 407.026 & 282.982 \\
			2 & 2 &  1.000 & 500.489 & 283.002 \\
			2 & 3 &  1.000 & 508.510 & 279.529 \\
			\hline 
			3 & 1 &  0.617 & 439.133 & 292.323 \\
			3 & 2 &  1.000 & 501.304 & 285.299 \\
			3 & 3 &  1.000 & 502.183 & 299.445 \\
			\hline 
			4 & 1 &  0.626 & 418.693 & 280.295 \\
			4 & 2 &  1.000 & 516.279 & 283.787 \\
			4 & 3 &  1.000 & 506.026 & 287.811 \\
			\hline 
			\hline 
			\multicolumn{5}{c}{$n = 2000$} \\ 
			\hline 
			1 & 1 &  0.633 & 839.780 & 550.367 \\
			1 & 2 &  1.000 & 1030.876 & 583.853 \\
			1 & 3 &  1.000 & 1013.996 & 598.250 \\
			\hline 
			2 & 1 &  0.634 & 817.342 & 548.831 \\
			2 & 2 &  1.000 & 992.001 & 595.322 \\
			2 & 3 &  1.000 & 980.492 & 577.158 \\
			\hline 
			3 & 1 &  0.679 & 845.336 & 559.742 \\
			3 & 2 &  1.000 & 1032.078 & 572.295 \\
			3 & 3 &  1.000 & 974.835 & 595.405 \\
			\hline 
			4 & 1 &  0.635 & 804.548 & 537.093 \\
			4 & 2 &  1.000 & 986.084 & 579.031 \\
			4 & 3 &  1.000 & 1010.196 & 574.338 \\
			\hline 
			\hline 
			\multicolumn{5}{c}{$n = 3000$} \\ 
			\hline 
			1 & 1 &  0.649 & 1242.521 & 858.306 \\
			1 & 2 &  1.000 & 1527.928 & 847.763 \\
			1 & 3 &  1.000 & 1460.141 & 871.720 \\
			\hline 
			2 & 1 &  0.626 & 1259.262 & 833.856 \\
			2 & 2 &  1.000 & 1485.588 & 862.315 \\
			2 & 3 &  1.000 & 1526.208 & 884.251 \\
			\hline 
			3 & 1 &  0.640 & 1227.862 & 843.167 \\
			3 & 2 &  1.000 & 1470.636 & 868.543 \\
			3 & 3 &  1.000 & 1524.807 & 866.474 \\
			\hline 
			4 & 1 &  0.648 & 1325.914 & 849.878 \\
			4 & 2 &  1.000 & 1514.407 & 883.120 \\
			4 & 3 &  1.000 & 1492.666 & 871.562 \\
			\hline 
			\hline 
		\end{tabular}
	\caption{Base setting results with $\theta_h = 0.80$, $nbd = 1$}
	\end{figure}
	\pagebreak

	\item[Interpretation] Not much improvements as compared to the "without replacement" (choiceUpdS=2) results. We know that the expected number of misses (theoretical) for the hide-and-seek without replacement is $\frac{n-1}{2}$.

\end{description}

\pagebreak

\section{Experiments with different neighborhood sizes} \label{neigh}

\begin{description}
	\item[Setting] The setting for this experiment is identical to that of the base experiments in section~\ref{base_expt}. However, the neighborhood size ($nbd$) of each box is varied $\{1,2,3\}$. This experiment is carried out to observe if the size of neighborhood has any effect on expected number of misses when the seeker update is of type-3 (hot-cold).
	
	\item[Result] The following table shows the effect of different $nbd$ on expected number of misses. This is only applicable to update of type-3 (hot-cold).
	\begin{figure}[!h]
		\centering
		\begin{tabular}{llll}
			\hline \hline
			choiceH & $nbd = 1$ & $nbd = 2$ & $nbd = 3$ \\
			\hline \hline 
			\multicolumn{4}{c}{$n = 1000$} \\ 
			\hline
			1 & 492.347 & 514.478 & 491.224 \\
			2 & 507.946 & 505.859 & 490.422 \\
			3 & 494.038 & 490.952 & 479.129 \\
			4 & 512.331 & 488.430 & 504.325 \\
			\hline
			\multicolumn{4}{c}{$n = 2000$} \\ 
			\hline
			1 & 1009.698 & 980.231 & 1018.029 \\
			2 & 1002.935 & 980.974 & 1003.716 \\
			3 & 998.342 & 976.232 & 982.563 \\
			4 & 974.992 & 995.226 & 1038.486 \\
			\hline
			\multicolumn{4}{c}{$n = 3000$} \\
			\hline
			1 & 1462.101 & 1537.008 & 1491.913 \\
			2 & 1449.057 & 1538.320 & 1517.777 \\
			3 & 1506.054 & 1509.251 & 1532.587 \\
			4 & 1486.285 & 1537.198 & 1500.848 \\
			\hline 
		\end{tabular}
		\caption{Effect of $nbd$ on mean misses ($\theta_h = 0.50$)}
	\end{figure}

	\begin{figure}[!h]
		\centering
		\begin{tabular}{llll}
			\hline \hline
			choiceH & $nbd = 1$ & $nbd = 2$ & $nbd = 3$ \\
			\hline \hline 
			\multicolumn{4}{c}{$n = 1000$} \\ 
			\hline
			1 & 498.926 & 485.493 & 486.835 \\
			2 & 508.510 & 502.662 & 485.133 \\
			3 & 502.183 & 495.954 & 488.814 \\
			4 & 506.026 & 501.933 & 500.365 \\
			\hline
			\multicolumn{4}{c}{$n = 2000$} \\ 
			\hline
			1 & 1013.996 & 1048.778 & 981.851 \\
			2 & 980.492 & 1037.643 & 1059.751 \\
			3 & 974.835 & 1006.264 & 1017.773 \\
			4 & 1010.196 & 995.741 & 1001.974 \\
			\hline
			\multicolumn{4}{c}{$n = 3000$} \\
			\hline
			1 & 1460.141 & 1500.349 & 1504.404 \\
			2 & 1526.208 & 1513.013 & 1546.749 \\
			3 & 1524.807 & 1518.873 & 1528.463 \\
			4 & 1492.666 & 1465.570 & 1533.059 \\
			\hline 
		\end{tabular}
		\caption{Effect of $nbd$ on mean misses ($\theta_h = 0.80$)}
	\end{figure}
	\pagebreak
	
	\item[Interpretation] Neighborhood does not show much effect.
\end{description}

\section{Experiments with Hot and cold thresholds: $\theta_h$, $\theta_c$}

\begin{description}
	\item[Setting] We thought of a setting where the seeker does not distribute its probability mass to its neighbor. This setting is more realistic, when we open a cold box. If we open a cold box then it is highly probable that the nearby boxes (its neighbors) are also cold. So, there is no point in distributing its mass to these cold neighbors. There we formalised a concept of a cold threshold ($\theta_c$) i.e. a performance threshold where we say that if the performance of a box $i$ is $\leq \theta_c$, then it is a cold box. We devise the following setting therefore:
	\begin{enumerate}
		\item Open a box $i$
		\item If $perf(i) \geq \theta_h$: distribute its probability mass to all the unopened boxes in its neighbors. 
		\item If $perf(i) \leq \theta_c$: distribute its probability mass to all the unopened boxes except its neighbors.
		\item If none of 2 or 3: distribute its probability mass to all the unopened boxes.
		\item Repeat 1--4 until the hider is found.
	\end{enumerate}
	We conduct this experiment for three different neighborhood sizes as disuseed in section~\ref{neigh}. The thresholds are fixed at $\theta_h = 0.80$ and $\theta_c = 0.4$.
	
	\item[Results] Here are the results:
	\begin{figure}[!h]
		\centering
		\begin{tabular}{lllll}
			\hline \hline 
			choiceH & choiceUpdS & SuccessRate & mean(misses) & sd(misses) \\
			\hline \hline 
			\multicolumn{5}{c}{$n = 1000$} \\ 
			\hline 
			1 & 1 &  0.647 & 415.765 & 279.076 \\
			1 & 2 &  1.000 & 485.743 & 291.479 \\
			1 & 3 &  1.000 & 483.394 & 280.817 \\
			\hline 
			2 & 1 &  0.622 & 424.259 & 295.290 \\
			2 & 2 &  1.000 & 505.753 & 288.841 \\
			2 & 3 &  1.000 & 497.388 & 285.391 \\
			\hline 
			3 & 1 &  0.614 & 425.721 & 268.688 \\
			3 & 2 &  1.000 & 501.925 & 291.147 \\
			3 & 3 &  1.000 & 498.737 & 285.369 \\
			\hline 
			4 & 1 &  0.629 & 398.860 & 277.083 \\
			4 & 2 &  1.000 & 497.892 & 296.853 \\
			4 & 3 &  1.000 & 511.561 & 286.341 \\
			\hline 
			\hline 
			\multicolumn{5}{c}{$n = 2000$} \\ 
			\hline 
			1 & 1 &  0.645 & 854.050 & 568.091 \\
			1 & 2 &  1.000 & 1006.014 & 562.589 \\
			1 & 3 &  1.000 & 997.183 & 570.448 \\
			\hline 
			2 & 1 &  0.612 & 855.533 & 570.626 \\
			2 & 2 &  1.000 & 1041.809 & 592.339 \\
			2 & 3 &  1.000 & 1024.389 & 593.691 \\
			\hline 
			3 & 1 &  0.612 & 847.077 & 569.229 \\
			3 & 2 &  1.000 & 998.326 & 565.699 \\
			3 & 3 &  1.000 & 1016.857 & 574.992 \\
			\hline 
			4 & 1 &  0.627 & 820.746 & 542.065 \\
			4 & 2 &  1.000 & 1013.802 & 583.904 \\
			4 & 3 &  1.000 & 1018.995 & 584.717 \\
			\hline 
			\hline 
			\multicolumn{5}{c}{$n = 3000$} \\ 
			\hline 
			1 & 1 &  0.627 & 1249.045 & 816.693 \\
			1 & 2 &  1.000 & 1509.410 & 852.052 \\
			1 & 3 &  1.000 & 1430.123 & 860.659 \\
			\hline 
			2 & 1 &  0.617 & 1255.799 & 812.271 \\
			2 & 2 &  1.000 & 1451.379 & 881.715 \\
			2 & 3 &  1.000 & 1500.294 & 867.022 \\
			\hline 
			3 & 1 &  0.617 & 1259.626 & 852.565 \\
			3 & 2 &  1.000 & 1513.932 & 873.289 \\
			3 & 3 &  1.000 & 1541.243 & 855.966 \\
			\hline 
			4 & 1 &  0.643 & 1249.299 & 844.649 \\
			4 & 2 &  1.000 & 1506.715 & 853.001 \\
			4 & 3 &  1.000 & 1501.686 & 880.258 \\
			\hline 
			\hline 
	\end{tabular}
	\caption{Results for $nbd = 1$}
	\end{figure}

	\begin{figure}[!h]
		\centering
		\begin{tabular}{lllll}
			\hline \hline 
			choiceH & choiceUpdS & SuccessRate & mean(misses) & sd(misses) \\
			\hline \hline 
			\multicolumn{5}{c}{$n = 1000$} \\ 
			\hline 
			1 & 1 &  0.616 & 426.451 & 294.068 \\
			1 & 2 &  1.000 & 497.421 & 288.863 \\
			1 & 3 &  1.000 & 495.362 & 282.265 \\
			\hline 
			2 & 1 &  0.647 & 426.195 & 289.582 \\
			2 & 2 &  1.000 & 506.020 & 288.400 \\
			2 & 3 &  1.000 & 510.959 & 288.912 \\
			\hline 
			3 & 1 &  0.597 & 411.358 & 288.290 \\
			3 & 2 &  1.000 & 483.829 & 285.909 \\
			3 & 3 &  1.000 & 496.718 & 291.624 \\
			\hline 
			4 & 1 &  0.647 & 404.587 & 273.281 \\
			4 & 2 &  1.000 & 498.959 & 297.005 \\
			4 & 3 &  1.000 & 494.474 & 291.465 \\
			\hline 
			\hline 
			\multicolumn{5}{c}{$n = 2000$} \\ 
			\hline 
			1 & 1 &  0.620 & 832.484 & 558.683 \\
			1 & 2 &  1.000 & 1019.508 & 579.080 \\
			1 & 3 &  1.000 & 963.953 & 574.337 \\
			\hline 
			2 & 1 &  0.616 & 842.664 & 577.997 \\
			2 & 2 &  1.000 & 1005.651 & 582.930 \\
			2 & 3 &  1.000 & 986.946 & 571.343 \\
			\hline 
			3 & 1 &  0.630 & 838.575 & 558.411 \\
			3 & 2 &  1.000 & 976.245 & 583.901 \\
			3 & 3 &  1.000 & 1008.720 & 587.234 \\
			\hline 
			4 & 1 &  0.634 & 844.882 & 557.914 \\
			4 & 2 &  1.000 & 990.474 & 575.303 \\
			4 & 3 &  1.000 & 1048.879 & 587.468 \\
			\hline 
			\hline 
			\multicolumn{5}{c}{$n = 3000$} \\ 
			\hline 
			1 & 1 &  0.607 & 1222.819 & 829.962 \\
			1 & 2 &  1.000 & 1461.000 & 864.485 \\
			1 & 3 &  1.000 & 1526.804 & 864.689 \\
			\hline 
			2 & 1 &  0.633 & 1249.983 & 828.909 \\
			2 & 2 &  1.000 & 1474.204 & 857.236 \\
			2 & 3 &  1.000 & 1520.048 & 851.266 \\
			\hline 
			3 & 1 &  0.654 & 1259.735 & 808.109 \\
			3 & 2 &  1.000 & 1496.244 & 875.656 \\
			3 & 3 &  1.000 & 1479.321 & 870.462 \\
			\hline 
			4 & 1 &  0.616 & 1223.940 & 866.136 \\
			4 & 2 &  1.000 & 1478.714 & 868.126 \\
			4 & 3 &  1.000 & 1491.112 & 867.572 \\
			\hline 
			\hline  
		\end{tabular}
		\caption{Results for $nbd = 2$}
	\end{figure}

	\begin{figure}[!h]
		\centering
		\begin{tabular}{lllll}
			\hline \hline 
			choiceH & choiceUpdS & SuccessRate & mean(misses) & sd(misses) \\
			\hline \hline 
			\multicolumn{5}{c}{$n = 1000$} \\ 
			\hline 
			1 & 1 &  0.607 & 419.806 & 284.285 \\
			1 & 2 &  1.000 & 506.118 & 288.646 \\
			1 & 3 &  1.000 & 500.512 & 286.805 \\
			\hline 
			2 & 1 &  0.621 & 427.678 & 290.978 \\
			2 & 2 &  1.000 & 480.902 & 297.833 \\
			2 & 3 &  1.000 & 492.826 & 282.865 \\
			\hline 
			3 & 1 &  0.625 & 401.525 & 277.746 \\
			3 & 2 &  1.000 & 510.823 & 288.126 \\
			3 & 3 &  1.000 & 500.524 & 289.540 \\
			\hline 
			4 & 1 &  0.648 & 432.190 & 285.944 \\
			4 & 2 &  1.000 & 499.628 & 287.115 \\
			4 & 3 &  1.000 & 500.301 & 294.824 \\
			\hline 
			\hline 
			\multicolumn{5}{c}{$n = 2000$} \\ 
			\hline 
			1 & 1 &  0.620 & 839.482 & 565.586 \\
			1 & 2 &  1.000 & 1020.989 & 574.616 \\
			1 & 3 &  1.000 & 969.406 & 577.838 \\
			\hline 
			2 & 1 &  0.618 & 824.794 & 542.836 \\
			2 & 2 &  1.000 & 997.555 & 580.826 \\
			2 & 3 &  1.000 & 999.403 & 583.367 \\
			\hline 
			3 & 1 &  0.628 & 813.912 & 551.922 \\
			3 & 2 &  1.000 & 980.755 & 567.618 \\
			3 & 3 &  1.000 & 990.786 & 577.004 \\
			\hline 
			4 & 1 &  0.636 & 842.928 & 572.518 \\
			4 & 2 &  1.000 & 1007.189 & 573.612 \\
			4 & 3 &  1.000 & 989.579 & 584.231 \\
			\hline 
			\hline 
			\multicolumn{5}{c}{$n = 3000$} \\ 
			\hline 
			1 & 1 &  0.636 & 1276.505 & 848.652 \\
			1 & 2 &  1.000 & 1485.448 & 870.575 \\
			1 & 3 &  1.000 & 1459.934 & 855.671 \\
			\hline 
			2 & 1 &  0.619 & 1200.735 & 843.910 \\
			2 & 2 &  1.000 & 1471.909 & 877.663 \\
			2 & 3 &  1.000 & 1572.459 & 862.723 \\
			\hline 
			3 & 1 &  0.623 & 1295.474 & 884.821 \\
			3 & 2 &  1.000 & 1473.981 & 869.361 \\
			3 & 3 &  1.000 & 1481.696 & 887.774 \\
			\hline 
			4 & 1 &  0.630 & 1239.613 & 840.033 \\
			4 & 2 &  1.000 & 1515.664 & 876.244 \\
			4 & 3 &  1.000 & 1467.864 & 864.726 \\
			\hline 
			\hline  
		\end{tabular}
		\caption{Results for $nbd = 3$}
	\end{figure}
	\pagebreak
	
	\item[Interpretation] Similar to the $\theta_h$ update in one of the earlier section. Not much improvements as compared to the "without replacement" (choiceUpdS=2) results. We know that the expected number of misses (theoretical) for the hi
de-and-seek without replacement is $\frac{n-1}{2}$.
\end{description}




%\section{Degeneracy variation Results}
%
%This section contains results for new settings (1) described above. The degeneracy was set at 25\% of $n$ in the base setting in the section \ref{setting:base}. The new settings for degeneracy is varied as: $\{0.10n, 0.20n, 0.30n, \ldots, 0.90n\}$ for the not-so-easy hider. It should be noted that when the degeneracy is reduced so much (viz. more boxes with probability mass greather than 0), the not-so-easy hider may not remain as it is. It could approach to be an pure non-uniform distribution (not-so-hard hider). We perform these experiments just with a fixed\footnote{Can be studied for any $n$. Just to avoid confusions, we fix $n$.} $n = 1000$.
%
%\textcolor{red}{----to be done----}
	
\end{document}
